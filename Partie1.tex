\chapter{Découverte du machine learning}
\section{Quelle est la différence entre apprentissage supervisé et non-supervisé ?}
Un apprentissage supervisé possède des données en entrée et en sortie. Ce type d’apprentissage possède un training set, un ensemble de données qui permet d’entraîner notre algorithme avec des données en entre à tester et les différentes solutions associées. Cela lui permet d’avoir une base pour ensuite étudier de nouveaux individus et d’avoir une idée de quelle type de solution l’algorithme doit chercher.\\

Pour l’apprentissage non-supervisé, il n’y a pas de donnée en sortie mais juste en entrée. C’est à l’algorithme de déduire les points importants et de proposer ses solutions sans avoir été entraîné auparavant.

\section{Quelle est la différence entre classification et régression ? Est-ce de l’apprentissage supervisé ou non-supervisé ?}
La classification et la régression sont 2 types d’algorithmes utilisant un apprentissage supervisé. On utilise la classification lorsque les solutions souhaitées sont des catégories comme des pommes ou des oranges. Alors que la régression est utilisée pour des valeurs numériques comme pour prédire le chiffre d’affaire d’une entreprise. La régression essaye de comprendre les relations entre les différentes variables.


\section{Qu’est-ce que du clustering ? Quelle est la différence avec la classification ?}
La principale différence entre la classification et le clustering est que la classification utilise un apprentissage supervisé alors que le clustering suit un apprentissage non-supervisé. Le clustering se base sur les similitudes des paires en entrée, et sur son expérience au fur et à mesure de de tester différentes entrée. Le temps d’exécution peut être très élevé si le nombre d’exemple en entrée est très élevé (plusieurs millions)

\newpage
\section{Sur kaggle ou driven-data, choisissez 5 exemples de compétitions et dites quel est le type de problème (classification, régression, clustering ou autre)}

\href{https://www.kaggle.com/competitions}{Liste des compétitions sur Kaggle}


\subsection{Compétition 1} 
\href{https://www.kaggle.com/competitions/house-prices-advanced-regression-techniques/data}{Compétition 1 sur la prédiction du prix d'une maison}\\

On remarque que le jeu de donnée possède un jeu de données d’entraînement et un jeu de donnée à tester. On est donc sur un apprentissage supervisé. Il faut trouver une valeur numérique pour prédire le prix de vente. Je pense qu’il faut utiliser une régression.


\subsection{Compétition 2} 
\href{https://www.kaggle.com/competitions/spaceship-titanic/overview}{Compétition 2 sur la prédiction de quel passager seront transportés dans une autre dimension}\\

On remarque que le jeu de donnée possède un jeu de données d’entraînement et un jeu de donnée à tester. On est donc sur un apprentissage supervisé. Il faut prédire si le passager va voyager dans une autre dimension ou non. Je pense qu’il faut utiliser une classification parce que c’est un choix binaire.


\subsection{Compétition 3} 
\href{https://www.kaggle.com/competitions/titanic/overview}{Compétition 3 sur la prédiction des survivants du Titanic}\\

On remarque que le jeu de donnée possède un jeu de données d’entraînement et un jeu de donnée à tester. On est donc sur un apprentissage supervisé. Il faut prédire si le passager du Titanic va mourir ou non. Je pense qu’il faut utiliser une classification parce que c’est un choix binaire.


\subsection{Compétition 4} 
\href{https://www.kaggle.com/competitions/nlp-getting-started}{Compétition 4 sur la prédiction de si le contenu d'un tweet est réel ou non}\\

On remarque que le jeu de donnée possède un jeu de données d’entraînement et un jeu de donnée à tester. On est donc sur un apprentissage supervisé. Il faut prédire si l’information dans le tweet est vrai ou non. Je pense qu’il faut utiliser une classification parce que c’est un choix binaire.


\subsection{Compétition 5} 
\href{https://www.drivendata.org/competitions/66/flu-shot-learning/page/211/}{Compétition 5 sur la prédiction de quel personne vont recevoir leusr doses de vaccin}\\

On remarque que le jeu de donnée possède un jeu de données d’entraînement et un jeu de donnée à tester. On est donc sur un apprentissage supervisé. Il faut prédire si n individu va se faire vacciner pour h1n1 ou/et pour la grippe saisonnière ou non. Je pense qu’il faut utiliser une classification parce que c’est un choix binaire, il faut classer les personnes en fonction de quelle vaccin ils vont choisir ou non.


\section{Quel type de problème de machine learning vous semble le plus répandu ?}
Je remarque déjà que le type d’apprentissage le plus utilisé est l’apprentissage supervisé. On en déduit donc 2 types de problèmes qui sont le plus répandu et qui sont la classification et la régression. D’après les exemples précédents, j’ai l’impression que le problème de machine learning le plus répandu est la classification.