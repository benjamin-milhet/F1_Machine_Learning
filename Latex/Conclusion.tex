\chapter{Difficultés rencontrées}
Ma principale difficulté rencontrée durant ce projet, mais aussi à travers l'ensemble de ce semestre est la sous-estimation du travail demandée dans certaines matières et la mauvaise gestion de mon temps libre pour mener à bien le projet de 4A. En effet, lors de la réalisation de ma première planification, je ne pensais pas avoir une charge de travail à la maison aussi importante pour la rédaction des différents rapports et la préparation aux partiels.\\ 

Une autre difficulté au début du projet était ma faible connaissance du langage Python et de ses librairies pour le Machine Learning. J'avais déjà un peu programmé en Python durant certain TP, mais c'est la première fois que je réalise un projet complet de plusieurs semaines avec ce langage. La réelle nouveauté fût la bibliothèque Scikit-Learn, très utilisé en Machine Learning. \\

\chapter{Points à améliorer}
Le premier point à améliorer et le plus important constaté durant ce projet est la gestion de mon temps libre. Pour remédier à cela, je pourrais essayer de planifier chacune de mes semaines ainsi que de faire des sessions travails sans divertissement possible tel que une heure de 30 minutes de travail puis 10 minutes de pause par exemple.\\

Un second point d'amélioration serait d'utiliser et comparer plus d'algorithme de régression comme la régression Ridge. De plus, ce serait aussi intéressant de comparer les tests avec d'autres pilotes de Formule 1 avec des types de données différentes. Par exemple un pilote qui n'a jamais gagné un grand-prix de sa carrière. \\

Un dernier point d'amélioration serait de chercher d'autre type de prédiction, comme par exemple tenter de prédire le temps d'un pilote pour un tour de qualification ou quelles écuries de Formule 1 finira première du championnat.



\chapter{Conclusion}
L'ensemble de ce projet m'a permis d'approfondir mes connaissances en Machine Learning et de découvrir différents types d'apprentissage comme l'apprentissage supervisé et non-supervisé. Ce projet permet de combiner la programmation, le Machine Learning et la Formule 1 afin de prédire le nombre de points qu'un pilote peut gagner par course. Nous avons réussi à mettre en évidence les variables les plus importantes pour prédire le nombre de points que va gagner un pilote de Formule 1 par course à l'aide de différents algorithmes tels que la régression linéaire, la régression Lasso et Ridge pour créer des modèles de prédiction.